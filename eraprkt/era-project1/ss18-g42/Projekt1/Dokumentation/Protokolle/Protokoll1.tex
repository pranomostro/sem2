\documentclass{article}
\usepackage[utf8]{inputenc}
\usepackage{blindtext}
\title{ERA-Projekt Protoll 1, 2018-04-27}
\author{Adrian Regenfuß}
\begin{document}
\maketitle

\section{Einrichtung eines Gitlab-Repository}

Bei dem ersten Treffen wurde zunächst ein separates Gitlab-Repository
für das Projekt eingerichtet, da die Teilnehmer des Praktikums mit Git
mehr Erfahrung vorweisen als mit Svn, und kopierte die Inhalte des
Svn-Repository auf das Gitlab-Repository.

\section{Verteilung der Rollen}

Daraufhin wurden die Rollen für die verschiedenen Projekte verteilt:
Beim Assembler-Praktikum wurde die Projektleitung an Adrian Regenfuß vergeben,
der Vortrag an Korbinian Stein und die Dokumentation an Till Müller.
Beim VHDL-Praktikum wurde die Projektleitung an Till Müller vergeben,
der Vortrag an Adrian Regenfuß und die Dokumentation an Korbinian Stein.

\section{Besprechung der Assembler-Aufgabe}

Daraufhin wurde die Lösung für die Assembler-Aufgabe diskutiert, wobei
die beiden in der Aufgabenstellung angedeuteten Lösungen in Betracht
gezogen wurden (als der Lookup-Table und die Reihenentwicklung). Hierbei
entschied man sich für den Lookup-Table, aufgrund der höheren Zeiteffizienz.

\end{document}
