\documentclass{article}
\usepackage[utf8]{inputenc}
\usepackage{blindtext}
\title{ERA-Projektleiterbericht Projekt 1, Assembler}
\author{Adrian Regenfuß}
\begin{document}
\maketitle

\section{Aufgabe des Projektes}

Das Ziel des Projektes bestand darin, den Arkussinus in x86-Assembler
mithilfel des Assemblers nasm zu implementieren. Hierbei sollten
zudem Tests für das Programm und eine ausführliche Dokumentation
der Ergebnisse verfasst und bereitgestellt werden und eine Organisationsstruktur
innerhalb des Teams hergestellt werden.

\section{Bestehende Kenntnisse}

Bei diesem Projekt konnten bestehende Kenntnisse besonders gut
eingesetzt werden. Neben den im vorigen Semester erlernten Grundlagen
für Assembler bestand bereits ein Wissen über die Programmiersprache
C und die Verwendung von Makefiles. Diese konnte eingesetzt werden, um
die Entwicklungsumgebung mit Tests und Makefiles schnell zu erstellen
und sich daraufhin größtenteils auf die Assembler-Implementierung zu
fokussieren. Bei Assembler bestand bereits grundlegendes Wissen über die
x86-Architektur, den Stack und Register.

Informationen über die FPU, den Aufruf von in Assembler geschriebenen
Funktionen aus C heraus und die C-Aufrufkonvention konnten ohne
größere Probleme über Recherche im Internet gefunden werden. Da
bereits alle Gruppenmitglieder grundlegende Fähigkeiten im Umgang mit
dem Versionskontrollsystem git hatten, aber nur geringe Kenntnisse über
svn hatten, wurde auf der Plattform Gitlab ein separates Repository
eingerichtet, über das der Großteil der Arbeit ausgetauscht wurde.

\section{Aufgabenverteilung und Organisation}

Team-interne Kommunikation und Organisation fanden größtenteils über
den Voice-Chat-Service Discord und eine eigens dafür erstellte WhatsApp
Gruppe statt. Ergebnisse wurden, wie bereits beschrieben, aufgrund
höherer Familiarität über ein GitLab Repository ausgetauscht.

Die Rollen wurden auf der Basis von Schätzungen der Kenntnisse und
Performance der Mitglieder verteilt, einzelne Aufgaben wurden meistens
freiwillig von den individuellen Mitgliedern übernommen, die sich am
kompetentesten für die gegebene Aufgabe hielten.

Wurde Hilfe benötigt, genügte meistens eine kurze Frage in der
WhatsApp-Gruppe oder im Discord-Channel. Treffen wurden anfangs
wöchentlich abgehalten, dies wurde dann jedoch zu späteren Zeitpunkten
weniger notwendig, da die Aufgabenverteilung im Projekt recht klar war
und größere Probleme schnell und unbürokratisch gelöst wurden. In
den Tagen vor einer Abgabe wurde meistens noch eine kurze gemeinsame
Besprechung der Ergebnisse und eine Kontrolle dieser abgehalten.

Kollaboration war aufgrund der Nutzung eines verteilten
Versionskontrollsystems einfach, es mussten jedoch mehrere Male Konflikte
in den Dateien gelöst werden.  Aufgrund der relativ hohen Modularität
des Projektes konnten unabhängige Teile paralell von verschiedenen
Mitgliedern bearbeitet werden. Zum Beispiel wurden die Tests paralell
zu den frühen Stadien der Implementierung verfasst, was einen Ablauf
vergleichbar mit dem sogenannten Test-Driven-Development erlaubte.

\section{Probleme und Lösungen}

Im Projekt ergaben sich einige Male kleiner Probleme, und im relativ
späten Stadium stellte sich ein größeres Problem heraus. Die kleineren
Probleme konnten durch kurze Absprachen gelöst werden.

Bei dem größeren Problem, das in der ungenügenden Genauigkeit der
Implementierung bestand, wurde ein Treffen über Discord vereinbart,
in dem mögliche Lösungsansätze in der Gruppe besprochen und analysiert
wurden. Man einigte sich auf drei kombinierte Lösungsstrategien:
Das Aufteilen der Interpolationstabelle in zwei Tabellen, wobei
eine für Eingabewerte unter 0.9 und eine für Eingabewerte über 0.9
verwendet wurde. Dies sollte die Präzision erhöhen, da der Arksinus
nah an 1 stärker zu wachsen beginnt. Zudem wurde beschlossen, die
Symmetrie des Arkussinus auszunutzen und nur positive Tabellenwerte
zu speichern, die notfalls bei negativen Eingabewerten negiert wurden.
Zuletzt wurden die Genauigkeitsbedingungen in den Tests auf ein
realistisches Maß gesenkt.

Das Auftreten eines größeren Problems wurde also innerhalb weniger Tage
durch eine Notfallsitzung angegangen und erfolgreich gelöst, was die
Entwicklung des Programmes nur um einen um einige Tage verlängerte.
Die stellte jedoch kein Problem dar, da der Entwicklungsplan eine
solche Verzögerung bereits einberechnet hatte.

\section{Fazit}

Die Leitung dieses Projektes war eine angenehme und kollaborative Aufgabe.
Zu keinem Zeitpunkt mussten Teammitglieder zur Arbeit aufgefordert werden,
häufig wurden Aufgaben selbstständig übernommen. Dennoch bestand zu jedem
Zeitpunkt ein reger Austausch zwischen den Teammitgliedern und hierarchische
Koordination war fast nie notwendig. Besonders Korbinian Stein tat sich durch
sein motiviertes und schnelles Arbeiten an der Implementierung des Programmes
hervor.

\end{document}
