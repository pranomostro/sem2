\documentclass{article}
\usepackage[utf8]{inputenc}
\usepackage{blindtext}
\title{ERA-Projekt Protoll 1, 2018-06-15}
\author{Adrian Regenfuß}
\begin{document}
\maketitle

\section{Ungenauigkeiten in der Assembler-Implementierung}

Bei diesem Treffen wurde bemerkt, dass die Assembler-Implementierung
in ihrere Genauigkeit nicht den in den Tests vorgegebenen Standards
entsprach. Verschiedene Vorschläge wurden umgesetzt, um diese Problem
zu lösen.

\section{Halbieren der Tabellengröße}

 Zunächst wurde vorgeschlagen, die Genauigkeit der Lookup-Tabelle
dadurch zu erhöhen, indem man die Symmetrie des Arkussinus ausnutzt.
Hierbei müssen nur die Ergebnisse für positive Werte gespeichert werden,
die notfalls bei negativen Eingabewerten negiert werden.

\section{Einführen einer Tabelle für Werte am oberen Rand des Wertebereiches}

Zudem wurde die Lookup-Tabelle in zwei Tabellen gleicher Größe aufgespalten,
eine für Eingabewerte unter 0.9, und eine für Eingabewerte über 0.9.
Die würde die Präzision erhöhen, da der Arkussinus sich bei größer werdenden
Eingabewerten immer stärker verändert.

\section{Lockern der Genauigkeitsbedingungen}

Die erste Version der Tests hatte geprüft, ob das Ergebnis bis auf die
sechste Nachkommastelle dem korrekten Ergebnis entsprach. Dies wurde gelockert,
so dass nur noch die ersten drei Nachkommastellen erfolgskritisch waren.

\end{document}
