\documentclass[11pt,a4paper]{article}

\usepackage[utf8]{inputenc}
\usepackage{graphicx}
\usepackage{float}
\usepackage{titlesec}
\usepackage[singlespacing]{setspace}
\usepackage[left=2.5cm,right=2.5cm,top=2cm,bottom=2cm]{geometry}

\setlength{\parskip}{5pt plus 1 pt minus 1 pt}
\titlespacing\section{0pt}{3pt}{3pt}
\titlespacing\subsection{3pt}{3pt}{3pt}

\begin{document}

\thispagestyle{empty}
\pagestyle{empty}

1. The dynamic model examines how a system reacts to events from the
outside, the object model deals with the structure of a system, and a
dynamic model describes how a system reacts to outside event.

2. Software development is difficult because it deals with systems that
increase in complexity very fast, and software developers have to deal
with domains they are unfamiliar with.

3. Change happens for various reasons. One is the sudden change of
requirements: A client realizes starts describing a feature differently,
thereby changing the requirements. Or a previously unknown behaviour
in the system is discovered, and has to be fixed in order to implement
another feature. Another change might be a change in the software
environment and the interfaces it provides to the system in development.
This environment can include libraries, protocols, operating systems
and other software systems. I once worked in a company here in munich
and spent 1 1/2 months fixing deprecation errors and warnings (a change
within the used libraries). This was a work due to a technological change.
During my work at Bosch-Rexroth I had to rewrite the app I had been
writing because of a previously inprecise specification. This is a
requirements change.

4. Modeling is the process of laying out the structure of a solution
for a given problem, programming is the process of implementing that
solution in source code.

5. A model is the structure of a system, a view is a collection of certain
aspects of a system, and a system is a collection of interconnected
parts in a certain structure.

\end{document}
