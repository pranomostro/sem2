\documentclass[11pt,a4paper]{article}

\usepackage[utf8]{inputenc}
\usepackage{graphicx}
\usepackage{float}
\usepackage{aeguill}
\usepackage{titlesec}
\usepackage[singlespacing]{setspace}
\usepackage[left=2.5cm,right=2.5cm,top=2cm,bottom=2cm]{geometry}

\setlength{\parskip}{5pt plus 1 pt minus 1 pt}
\titlespacing\section{0pt}{3pt}{3pt}
\titlespacing\subsection{3pt}{3pt}{3pt}

\begin{document}

\begin{center}
{\Large Homework Sheet 10}
\end{center}

\thispagestyle{empty}
\pagestyle{empty}

\section*{1}

\subsection*{a}

Bob might not have commited his changes, but only staged them. The
solution would be to commit and push the changes to the server.

Bob might have pushed his changes to a server different from the company
git server. The company would have to find that server, ask the owners
to delete their code from that server, push Bobs changes to the company
server and then fire Bob.

Bob might have worked on another branch and not merge the branch before
pushing. Then he would have to merge the new branch into master, pull from
the company server, resolve merge conflicts and then push the conflict-resolving
changes to the company server again.

\section*{2}

A merge conflict happens when two developers work on the same sections
in the same files and then try to push their changes to the same server.
The developer who pushes later then has to pull the changes from the
server and resolve the conflicts (often using both the git standard
merging algorithm and resolving the merge conflict by hand. The merge is
then commited and pushed to the server. Such conflicts can be avoided
by committing related changes (working on one module at a time) and
committing, pulling and pushing often to have only small conflicts when
they arise at all.

\end{document}
