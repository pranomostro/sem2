\documentclass[11pt,a4paper]{article}

\usepackage[utf8]{inputenc}
\usepackage{graphicx}
\usepackage{float}
\usepackage{aeguill}
\usepackage{titlesec}
\usepackage[singlespacing]{setspace}
\usepackage[left=2.5cm,right=2.5cm,top=2cm,bottom=2cm]{geometry}

\setlength{\parskip}{5pt plus 1 pt minus 1 pt}
\titlespacing\section{0pt}{3pt}{3pt}
\titlespacing\subsection{3pt}{3pt}{3pt}

\begin{document}

\begin{center}
{\Large Homework Sheet 05}
\end{center}

\thispagestyle{empty}
\pagestyle{empty}

\section*{1}

\subsection*{Advantage: Higher motivation because of high flexibility}

Programmers tend to favor high flexibility in their work, because it
means they can choose adjust their work to their state of mind. In the
SCRUM model, a programmer can for example choose to work on an easier,
but more monotonous process on one day when she is feeling a bit tired,
and on a complex and difficult problem one other day when she is feeling
more motivated and curious.

\subsection*{Advantage: Low hanging fruit are picked earlier}

Since easier backlog items will be picked and implemented earlier, the
product will be more fully implemented at a point in time than in another
model (hard problems are addressed at the end of the development process,
and can easily be abandoned if they are deemed impossible or harder than
previously estimated). For example a cooking app will first introduce
a rating system for recipes before a method of identifying already cooked
meals, before finding out the latter is nearly impossible. The first feature
is already implemented and shipped.

\subsection*{Advantage: Confusion and problems are noted early}

Because of the daily sprints, confusion and problems with the design are
recognized early and corrected before they cause any bigger damage.
For example, when a developer realizes that the network protocol the
project is supposed to use for file transmission is unencrypted
and therefore not usable, the whole team can know that within 24 hours
and react accordingly, for example switching to another protocol.

\subsection*{Disadvantage: Problems with the estimation of the duration of a task}

According to Hofstadter law "Everything takes longer than you estimate,
even if you consider Hofstadters law". This holds especially true
for software engineering. SCRUM ignores that and assumes that one
can reasonably estimate the time it takes to finish a sprint/backlog
task, since backlog items are often interdependent. For example, an
inexperienced developer might pick a task early in the process that many
other tasks depend on, but later realizes it is very hard and starts
blocking the rest of the project.

\subsection*{Disadvantage: The small team size makes it unsuitable for big projects}

Very large and complex systems can't be adequately implemented using
SCRUM, since the SCRUM team consists of 5-6 people, not enough for
systems that would require dozens of developers minimum. For example,
a system for train control can't be implemented using SCRUM, since it
is too big and complex. One might consider breaking the system into
subsystems that are then each implemented using SCRUM, but that results
in inconsistency between the different subsystems (since SCRUM teams
are quite tight and independently working from other teams).

\subsection*{The daily meetings disrupt the programmers flow}

Programmers need time to get into a state commonly called flow,
in which they are thinking about the problem at hand and building
models to solve the item at hand. Getting into flow takes time,
and a 10 minute daily sprint completely destroys flow. For example
a sprint at 2pm makes it impossible to get anything productive done
in the afternoon, since a programmer can't get into flow when he
knows that he will get interrupted in an hour and a half anyway,
and can't get into flow afterwards, since the day is already nearly
over.

\section*{2}

Difference: In the Unified Process, team sizes and structures change,
while in SCRUM the team stays the same during the development process.

Difference: In SCRUM, there is a lot of focus on the daily and weekly
sprint meetings, while in the Unified Process, there is an Assessment
every iteration, but it does not happen daily.

Similarity: Both models support some initialization, in SCRUM this is
the Kickoff-meeting, where the first product backlog is developed, in
the Unified Process this is the Engineering Stage, where the design of
the system is developed.

Similarity: Both circular processes are divided into minor and major
iterations, in the Unified Process this being the iteration and the phase,
producing minor and major milestones, in SCRUM being the daily sprints
and the 2-3 week iterations, the first consuming the currently worked
on items, the second consuming the sprint backlog.

\end{document}
