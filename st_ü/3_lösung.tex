\documentclass[11pt,a4paper]{article}

\usepackage[utf8]{inputenc}
\usepackage{graphicx}
\usepackage{float}
\usepackage{titlesec}
\usepackage[singlespacing]{setspace}
\usepackage[left=2.5cm,right=2.5cm,top=2cm,bottom=2cm]{geometry}

\setlength{\parskip}{5pt plus 1 pt minus 1 pt}
\titlespacing\section{0pt}{3pt}{3pt}
\titlespacing\subsection{3pt}{3pt}{3pt}

\begin{document}

\thispagestyle{empty}
\pagestyle{empty}

1.

\begin{itemize}
\item There must be a player car
\item There must be adversarial cars
\item The player must be able to move his car with the mouse
\item Cars need to be able to collide with each other
\item After a collision, one of the colliding cars should stop moving
\item The Player should be able to stop and start the game
\item The game should have music playing in the background
\item The game should play a sound when two cars crash
\end{itemize}

2.

\begin{itemize}
\item The game should run smoothly (i.e. no noticable slow-downs or speed issues)
\item The game should not crash
\item The source code for the game should be maintainable
\item The game should be compatible with multiple platforms
\item User friendliness
\end{itemize}

Additional category: Aesthetics, Portability

3.

As-is scenario: The user opens the game, klicks the start button, and then,
attempts to crash as many other cars on the board as possible. After failing to
do this for a few times, he gets increasingly angry and changes his cars speed,
making it much higher. This gives him a crucial advantage, which he uses to
defeat the other cars. Having accomplished this, he gets bored and continues
finishing his homework.

\vspace{0.5cm}

Visionary scenario: The player opens the game and starts the first
level. After having defeated half of the cars, the player car gets
a significant speed boost, which allows him to defeat the rest of the
cars. Having collided with all of them, the screen is cleared and a truck
chasing the player car appears. Being able to evade this truck and finally
colliding with it from the right, the player manages to succeed to the
next level. He then proceeds to waste his entire afternoon with the game.

4.

\begin{tabular}{l | p{100mm}}
Use case name & Car crash \\
\hline
Participating actors & Player Car, NPC Car \\
\hline
Entry conditions & The Player Car and the NPC Car have collided \\
\hline
Flow of events & After both cars have collided, the car crash sound is played. Subsequently the car having collided from the left will stop moving. \\
\hline
Exit conditions & If the player car was hit from the right, the player loses the game. Otherwise, the player continues the game and the NPC car is halted. \\
\hline
Special requirements & No car has been crunched beforehand. \\
\end{tabular}

5.

Extends signifies that something belongs to a certain class, and
if this relationship does not exist anymore, the object is transformed in
its entirety.  The Includes relationship is not this essential, but
rather accidental, that means if the relationship doesn't hold anymore,
this still doesn't fundamentally change the classes of the objects being
talked about. For example in the Bumpers game, the GameBoard includes
several cars, but would still be a GameBoard if it didn't, but an OldCar
extends a Car.

6.

Aggregation is a collection of objects into one shared entity. Composition
is a special form of aggregation, in which the parts don't exist on their
own, but just as parts of the aggregated entity. One example for aggregation
in Bumpers is the aggregation of NPC cars, but the GameBoard is a composition
of this group of cars and the player car.

\end{document}
